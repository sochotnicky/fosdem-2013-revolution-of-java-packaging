\RHpresentationHead{
  \documentclass[pdftex,unicode,xcolor=table]{beamer}
}

\RHarticleHead{
  % This does not work, because of colors, \insertauthor, etc.
  \documentclass[a4paper,12pt,pdftex,unicode]{article}
  \usepackage[envcountsect]{beamerarticle}
}



\mode<presentation> {
  \usetheme{Fedora}
  \setbeamertemplate{navigation symbols}{}
  \setbeamercovered{transparent=5}
}
\mode<article> {
  \usepackage{fullpage}
}

\mode<handout> {
  \usepackage{pgfpages}
  \pgfpagesuselayout{4 on 1}[a4paper,landscape,border shrink=5mm]
}


\usepackage{beamerredhat}
\usepackage{etex}
\usepackage[utf8]{inputenc}
%\usepackage[lang]{babel}
\usepackage{setspace,amsfonts,calc,upquote,hyperref,floatflt,graphicx}
\usepackage[table]{xcolor}
\usepackage{colortbl}
\usepackage[absolute,overlay]{textpos}\textposquirk


% presentation title/author/etc.
\title{(R)evolution of Java packaging in GNU/Linux}
\subtitle{Automating packaging}
\author{Authors: \\
  \em{Stanislav Ochotnický} sochotnicky@redhat.com\\
  \em{Mikołaj Izdebski} mizdebsk@redhat.com}
\date{Date: \em{2nd February 2013}}


% fancy section/part pages?
% \fancySectionOpens
% \fancyPartOpens

\begin{document}


% title pages
\mode<article> {
  \maketitle
  \newpage
}

\begin{rhbg}
  \begin{frame}
    \titlepage
    \begin{abstract}
      Over past 2 years, tooling and guidelines for packaging Java have changed in
      Fedora Linux considerably. What used to be a 1000 line build script can soon
      become 100 lines of mostly metadata. But all of that relies on sane build system
      with predictable behavior on Java side: Maven.
    \end{abstract}
    \note{
      Here be dragons. Each frame should have a note describing what we'll
      approximately talk about. Not word-for-word, just a generic idea
    }
  \end{frame}
\end{rhbg}


\section{Overview}
\Large
\begin{frame}
  \frametitle{Generic Java Packaging Problems (GJPP)}
  \begin{itemize}
  \item Lorem ipsum
  \end{itemize}
  \note{Note!}
\end{frame}

\section{History lessons}
\begin{frame}
  \frametitle{Original recipes}
  \begin{itemize}
  \item Lorem ipsum addendum
  \end{itemize}
  \note{Note!}
\end{frame}


\section{Future on the go}
\begin{frame}
  \frametitle{What to do}
  \note{We should add notes here and there
  }
\end{frame}

\section{General Tips}
\begin{frame}[fragile] % needed for verbatim blocks on slides
  \frametitle{Spec example}
  \begin{block}{Example of cruft}
    \scriptsize
\begin{verbatim}
Requires(post):    jpackage-utils >= 0:1.7.2
Requires(postun):  jpackage-utils >= 0:1.7.2
\end{verbatim}
  \end{block}
  \note{Note!}
\end{frame}



\begin{frame}
  \frametitle{Few more examples}
  \begin{block}{Itemize block 1}
    \begin{itemize}
    \item Z in X.Y.Z
    \end{itemize}
  \end{block}

  \begin{block}{Example block 2}
    \begin{itemize}
    \item A in ABC
    \end{itemize}
  \end{block}
  \note{Note!}
\end{frame}

\begin{frame}
  \frametitle{Showcase of different block}
  \begin{exampleblock}{Example block is a bit different}
    Oh is it really?
  \end{exampleblock}

  \begin{block}{How different?}
    Yes, really
  \end{block}
  \note{Note!}
\end{frame}


\begin{frame}
  \frametitle{Summary}
  \begin{itemize}
    \item Smart overview of previous stuff
  \end{itemize}
\end{frame}


\mode<presentation> {
  \Rhbg{\frame{\theend}}
}

\end{document}
